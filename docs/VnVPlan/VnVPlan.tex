\documentclass[12pt, titlepage]{article}

\usepackage{booktabs}
\usepackage{tabularx}
\usepackage{hyperref}
\hypersetup{
    colorlinks,
    citecolor=blue,
    filecolor=black,
    linkcolor=red,
    urlcolor=blue
}
\usepackage[round]{natbib}

%% Comments

\usepackage{color}

\newif\ifcomments\commentstrue %displays comments
%\newif\ifcomments\commentsfalse %so that comments do not display

\ifcomments
\newcommand{\authornote}[3]{\textcolor{#1}{[#3 ---#2]}}
\newcommand{\todo}[1]{\textcolor{red}{[TODO: #1]}}
\else
\newcommand{\authornote}[3]{}
\newcommand{\todo}[1]{}
\fi

\newcommand{\wss}[1]{\authornote{blue}{SS}{#1}} 
\newcommand{\plt}[1]{\authornote{magenta}{TPLT}{#1}} %For explanation of the template
\newcommand{\an}[1]{\authornote{cyan}{Author}{#1}}

%% Common Parts

\newcommand{\progname}{Damped Harmonic Ocsillator} % PUT YOUR PROGRAM NAME HERE
\newcommand{\authname}{
\\ Muhammad Waqar Ul Hassan Awan
} % AUTHOR NAMES                  

\usepackage{hyperref}
    \hypersetup{colorlinks=true, linkcolor=blue, citecolor=blue, filecolor=blue,
                urlcolor=blue, unicode=false}
    \urlstyle{same}
                                


\begin{document}

\title{\progname: Damped Harmonic Oscillator Illustrated by Online Calculator} 
\author{\authname}
\date{\today}
	
\maketitle

\pagenumbering{roman}

\section*{Revision History}

\begin{tabularx}{\textwidth}{p{3cm}p{2cm}X}
\toprule {\bf Date} & {\bf Version} & {\bf Notes}\\
\midrule
19 Feb, 2024 & 1.0 & Initial VnV document\\
\bottomrule
\end{tabularx}

\newpage

\tableofcontents

\newpage
\pagenumbering{arabic}

\section{Symbols, Abbreviations, and Acronyms}

As part of the comprehensive Verification and Validation (V\&V) Plan for this project, in 
order to ensure clarity and consistency in the use of symbols, abbreviations, and 
acronyms, we direct all project stakeholders and team members to the Software Requirements 
Specification (SRS) document. Specifically, Section 1 of the SRS document, along with its 
Subsections 1.2 and 1.3, provide a thorough definition and explanation of all symbols, 
abbreviations, and acronyms employed throughout the project documentation and 
implementation. 

\newpage

This document outlines the Verification and Validation (VnV) Plan for a 
project focused on simulating the dynamics of damped harmonic oscillators. It provides 
a comprehensive framework for ensuring the software's correctness, reliability, and 
usability in modeling oscillatory systems under various damping conditions. The plan 
covers a range of testing strategies, from input validation and functionality tests to 
non-functional assessments such as usability and maintainability. Additionally, it 
details the approach for aligning software implementation with the outlined requirements, 
ensuring a thorough evaluation across different operating systems. This roadmap guides the 
project's VnV efforts, aiming for high accuracy and user satisfaction.

\section{General Information}

\subsection{Summary}

The software being tested is named ``\progname(DHO)''. It is designed to simulate the 
dynamics of harmonic oscillators subjected to both linear and non-linear damping forces. 
DHO offers a detailed representation of oscillatory systems through  numerical solutions 
to differential equations, providing insights into their behavior under various damping 
conditions.

\subsection{Objectives}

The primary objective is to build confidence in the software's correctness and ensure its 
reliability in modeling damped oscillatory systems accurately. Another crucial goal is 
to demonstrate the software's adequacy in usability for educational and research purposes 
in physics and engineering. 
Following are some of the objectives that are out of scope:

\begin{itemize}
  \item We will not verify the quality of usability in terms of user interface design 
  beyond basic functionality and accessibility. This is due to limited resources and the 
  project's primary focus on the accuracy of simulation results.
  \item The project will assume that all external libraries used for numerical 
  computations and graphical representations are already verified by their respective 
  development teams. This assumption allows us to concentrate our resources on the core 
  functionalities of our software.
\end{itemize}

\subsection{Relevant Documentation}

Relevant documentation includes the Software Requirement Specification (SRS) document, 
which outlines the functional and non-functional requirements of the project. 
Additionally, design documents, such as the Module Guide (MG) and Module Interface 
Specification (MIS), are crucial as they detail the software's architecture and 
interaction interfaces. These documents are relevant to the Verification and Validation 
(VnV) efforts as they provide a foundation for testing strategies, ensuring that every 
aspect of the software is tested against the specified requirements and designs. By 
aligning VnV activities with these documents, we ensure a comprehensive evaluation of 
the software's functionality, reliability, and usability.

\section{Plan}

This section details the verification and validation (VnV) plan for the harmonic 
oscillator simulation project, designed and implemented as an individual, focusing on 
backend calculations and REST API functionality using Python/Flask, with ReactJS for the 
frontend.

\subsection{Verification and Validation Team}

As the sole contributor to this project, I will be responsible for all testing efforts. 
To enhance the robustness of the VnV process, I will take the help of two peers for 
review purposes: a primary reviewer and a secondary reviewer. These reviewers will 
provide feedback and insights to ensure the software's quality and checking if the 
requirements are met. Their roles will involve critiquing the design, testing strategy, 
and overall identification of any potential issues or areas for improvement in the project.

\subsection{SRS Verification Plan}

The SRS verification will involve a detailed review against a comprehensive checklist I 
will create, ensuring all functional and non-functional requirements are met. This 
checklist will be based on the specifications detailed in the SRS document. The SRS 
verification will also involve a review process, with the inclusion of a primary and a 
secondary reviewer.

\subsection{Design Verification Plan}

Given the individual nature of this project, design verification will focus on ensuring 
that the software's architecture and system design align with the outlined specifications 
and best practices for Python/Flask and ReactJS applications. The design verification will 
also incorporate a dual-peer review process, involving a primary and a secondary reviewer, 
to ensure a thorough examination of the software's architecture and system design.

\subsection{Verification and Validation Plan Verification Plan}

This step will involve a critical review of the VnV plan itself to ensure its adequacy 
and comprehensiveness. Adjustments will be made based on findings to improve the overall 
VnV strategy. The verification and validation (VnV) plan itself will undergo a review 
process involving both a primary and a secondary peer reviewer.

\subsection{Implementation Verification Plan}

The Implementation Verification Plan will include dynamic testing with unit tests 
primarily utilizing the PyTest framework for the Python/Flask backend. PyTest is chosen 
for its extensive support for testing Python applications, allowing for efficient testing 
of individual components and integration points. Static verification methods will 
include code walkthroughs, code inspections, and the use of static analyzers like Flake8 
or Pylint to ensure code quality and adherence to coding standards. This blend of dynamic 
and static testing ensures a thorough validation of the software's functionality and 
reliability.

\subsection{Automated Testing and Verification Tools}

For automated testing, the PyTest framework will be used for unit testing the Python/Flask 
components, due to its robustness and ease of use. For static code analysis, Flake8 will 
serve as the primary linter, ensuring code quality and adherence to Python coding 
standards. GitHub Actions will automate the CI/CD pipeline, running tests and linting on 
every push to ensure continuous quality. Code coverage metrics will be summarized using 
the coverage.py tool, providing insights into untested parts of the codebase to improve 
test completeness.

\subsection{Software Validation Plan}

For software validation, I will compare the simulation results with those from external 
tools such as the GeoGebra damped harmonic oscillator calculator and other reputable 
online calculators to ensure accuracy. This external validation serves as an effective 
method for confirming the software's correctness against established benchmarks. 
Additionally, task-based inspections and peer reviews will be utilized to verify that 
the requirements document captures the correct requirements, providing a comprehensive 
validation approach from both a technical and user-experience perspective.

\section{System Test Description}
	
\subsection{Tests for Functional Requirements}

Given the project's focus on accurately modeling and simulating damped harmonic 
oscillators, the testing strategy is organized around key functional areas derived from 
the SRS document.

\subsubsection{Input Parameters Handling}

\begin{enumerate}

\item{test-id1\\}

Control: Manual
					
Initial State: Application loaded with no data entered.
					
Input: Varying valid and invalid values for each parameter.
					
Output: Acceptance of valid inputs and rejection or error messages for invalid inputs 
according to constraints in SRS section 4.2.6.

Test Case Derivation: Verify input parameters for mass, spring constant, damping 
coefficient, initial displacement, and velocity.
					
How test will be performed: Manually enter data into the software and observe behavior.

\end{enumerate}

\subsubsection{Display and Confirmation of Entered Values}

\begin{enumerate}

\item{test-id1\\}

Control: Manual
					
Initial State: Parameters entered and awaiting confirmation.
					
Input: Parameters for simulation.
					
Output: Accurately display entered values for user review before simulation starts.

Test Case Derivation: Ensure entered values are correctly displayed for user confirmation.
					
How test will be performed: Check if the UI reflects the exact input values given by the user.

\end{enumerate}


\subsubsection{Calculation of Oscillator Dynamics}

\begin{enumerate}

\item{test-id1\\}

Control: Automated
					
Initial State: Inputs validated and simulation ready.
					
Input: Test scenarios with known outcomes.
					
Output: Calculated values match expected theoretical or empirical results within defined 
tolerances.

Test Case Derivation: Accuracy of displacement, velocity, and energy calculations over time.
					
How test will be performed: Use automated scripts to compare software output against 
benchmark cases.

\end{enumerate}

\subsubsection{Verification of Solution Correctness}

\begin{enumerate}

\item{test-id1\\}

Control: Manual
					
Initial State: Simulation completed with results generated.
					
Input: Values for comparison.
					
Output: Conformance to benchmarks with detailed reports on discrepancies.

Test Case Derivation: Solution verification against known values.
					
How test will be performed: Manual comparison to validate output against established 
benchmarks.

\end{enumerate}


\subsubsection{Output Presentation}

\begin{enumerate}

\item{test-id1\\}

Control: Manual
					
Initial State: Simulation executed with results available.
					
Input: User request for output display.
					
Output: Outputs (displacement, velocity, energy) are presented in an easily interpretable 
format (graphs, tables).

Test Case Derivation: Clarity and understandability of output presentation.
					
How test will be performed: User evaluation of output presentation for clarity and 
completeness.

\end{enumerate}

\subsection{Tests for Nonfunctional Requirements}

\subsubsection{Accuracy Testing}

\begin{enumerate}

\item{test-id1\\}

Type: Manual
					
Initial State: System initialized with predefined parameters for a damped oscillator.
					
Input: Specific sets of input parameters for mass, spring constant, and damping 
coefficient.
					
Output: Calculated displacement, velocity, and acceleration over time.
					
How test will be performed: Compare the software output with analytical solutions 
numerical simulations to calculate relative error.

\item{test-id2\\}

Type: Automated
					
Initial State: System at rest with no initial input.
					
Input/Condition: Range of damping coefficients representing underdamped, critically 
damped, and overdamped scenarios.
					
Output/Result: System's response time to reach equilibrium or a steady state.
					
How test will be performed: Use automated testing tools to run simulations and verify 
accuracy against theoretical expectations, reporting relative errors.

\end{enumerate}

\subsubsection{Usability Testing}

\begin{enumerate}

\item{test-id1\\}
  
Type: Manual, User Survey
          
Initial State: Users are provided with the software and basic instructions.
          
Input: Tasks requiring users to input parameters, run simulations, and interpret results.
          
Output: User feedback on ease of use, intuitiveness of the interface, and clarity of the 
results presentation.
          
How test will be performed: Conduct a usability test followed by a survey to gather 
qualitative and quantitative data from users.
  
\end{enumerate}


\subsubsection{Maintainability Testing}

\begin{enumerate}

\item{test-id1\\}
  
Type: Manual
          
Initial State: Existing software codebase and documentation.
          
Input: Simulated changes to models and damping functions.
          
Output: Effort required for implementation.
          
How test will be performed: Review and analysis by development team to estimate the 
effort needed for future modifications, ensuring it does not exceed 25\% of the 
original development effort
  
\end{enumerate}


\subsubsection{Portability Testing}

\begin{enumerate}

\item{test-id1\\}
  
Type: Automated/Manual
          
Initial State: Software ready for deployment.
          
Input: Operation on Windows 10 and above, macOS Catalina and above, and popular Linux 
distributions such as Ubuntu 20.04 LTS.
          
Output: Software runs without issues.
          
How test will be performed: Run the software on various operating systems and Browsers 
and verify seamless operation, as outlined in the VnV Plan
  
\end{enumerate}

\newpage

\subsection{Traceability Between Test Cases and Requirements}

\subsubsection{Function Requirements}

\begin{tabularx}{\textwidth}{p{3cm}p{2cm}X}
\toprule {\bf Requirement ID} & {\bf Test Case ID} &{\bf Description}\\
\midrule

R1 & test-id1 & Verifies the software's ability to accept input parameters for mass, 
spring constant, damping coefficient, and initial conditions.\\

R2 & test-id2 & Ensures entered values are correctly displayed for user confirmation.\\

R3 & test-id3 & Tests the accuracy of displacement, velocity, and energy calculations 
over time.\\

R4 & test-id4 & Verifies the correctness of solutions against known benchmarks.\\

R5 & test-id5 & Assesses the clarity and understandability of the software's output 
presentation.\\

\bottomrule
\end{tabularx}

\subsubsection{Non-Function Requirements}

\begin{tabularx}{\textwidth}{p{3cm}p{2cm}X}
\toprule {\bf Requirement ID} & {\bf Test Case ID} &{\bf Description}\\
\midrule

NFR1 & test-id1 & Validates computational accuracy against theoretical values to 
within 0.10\%.\\

NFR1 & test-id2 & Evaluates software usability through user surveys and usability 
tests.\\

NFR1 & test-id3 & Assesses maintainability by estimating effort required for future 
modifications and updates.\\

NFR1 & test-id4 & Checks software portability across Windows, macOS, and popular 
Linux distributions.\\

\bottomrule
\end{tabularx}

% \section{Unit Test Description}

% \wss{This section should not be filled in until after the MIS (detailed design
%   document) has been completed.}

% \wss{Reference your MIS (detailed design document) and explain your overall
% philosophy for test case selection.}  

% \wss{To save space and time, it may be an option to provide less detail in this section.  
% For the unit tests you can potentially layout your testing strategy here.  That is, you 
% can explain how tests will be selected for each module.  For instance, your test building 
% approach could be test cases for each access program, including one test for normal behaviour 
% and as many tests as needed for edge cases.  Rather than create the details of the input 
% and output here, you could point to the unit testing code.  For this to work, you code 
% needs to be well-documented, with meaningful names for all of the tests.}

% \subsection{Unit Testing Scope}

% \wss{What modules are outside of the scope.  If there are modules that are
%   developed by someone else, then you would say here if you aren't planning on
%   verifying them.  There may also be modules that are part of your software, but
%   have a lower priority for verification than others.  If this is the case,
%   explain your rationale for the ranking of module importance.}

% \subsection{Tests for Functional Requirements}

% \wss{Most of the verification will be through automated unit testing.  If
%   appropriate specific modules can be verified by a non-testing based
%   technique.  That can also be documented in this section.}

% \subsubsection{Module 1}

% \wss{Include a blurb here to explain why the subsections below cover the module.
%   References to the MIS would be good.  You will want tests from a black box
%   perspective and from a white box perspective.  Explain to the reader how the
%   tests were selected.}

% \begin{enumerate}

% \item{test-id1\\}

% Type: \wss{Functional, Dynamic, Manual, Automatic, Static etc. Most will
%   be automatic}
					
% Initial State: 
					
% Input: 
					
% Output: \wss{The expected result for the given inputs}

% Test Case Derivation: \wss{Justify the expected value given in the Output field}

% How test will be performed: 
					
% \item{test-id2\\}

% Type: \wss{Functional, Dynamic, Manual, Automatic, Static etc. Most will
%   be automatic}
					
% Initial State: 
					
% Input: 
					
% Output: \wss{The expected result for the given inputs}

% Test Case Derivation: \wss{Justify the expected value given in the Output field}

% How test will be performed: 

% \item{...\\}
    
% \end{enumerate}

% \subsubsection{Module 2}

% ...

% \subsection{Tests for Nonfunctional Requirements}

% \wss{If there is a module that needs to be independently assessed for
%   performance, those test cases can go here.  In some projects, planning for
%   nonfunctional tests of units will not be that relevant.}

% \wss{These tests may involve collecting performance data from previously
%   mentioned functional tests.}

% \subsubsection{Module ?}
		
% \begin{enumerate}

% \item{test-id1\\}

% Type: \wss{Functional, Dynamic, Manual, Automatic, Static etc. Most will
%   be automatic}
					
% Initial State: 
					
% Input/Condition: 
					
% Output/Result: 
					
% How test will be performed: 
					
% \item{test-id2\\}

% Type: Functional, Dynamic, Manual, Static etc.
					
% Initial State: 
					
% Input: 
					
% Output: 
					
% How test will be performed: 

% \end{enumerate}

% \subsubsection{Module ?}

% ...

% \subsection{Traceability Between Test Cases and Modules}

% \wss{Provide evidence that all of the modules have been considered.}
				
% \bibliographystyle{plainnat}

% \bibliography{../../refs/References}

% \newpage

% \section{Appendix}

% This is where you can place additional information.

% \subsection{Symbolic Parameters}

% The definition of the test cases will call for SYMBOLIC\_CONSTANTS.
% Their values are defined in this section for easy maintenance.

% \subsection{Usability Survey Questions?}

% \wss{This is a section that would be appropriate for some projects.}

% \newpage{}
% \section*{Appendix --- Reflection}

% The information in this section will be used to evaluate the team members on the
% graduate attribute of Lifelong Learning.  Please answer the following questions:

% \newpage{}
% \section*{Appendix --- Reflection}

% \wss{This section is not required for CAS 741}

% The information in this section will be used to evaluate the team members on the
% graduate attribute of Lifelong Learning.  Please answer the following questions:

% \begin{enumerate}
%   \item What knowledge and skills will the team collectively need to acquire to
%   successfully complete the verification and validation of your project?
%   Examples of possible knowledge and skills include dynamic testing knowledge,
%   static testing knowledge, specific tool usage etc.  You should look to
%   identify at least one item for each team member.
%   \item For each of the knowledge areas and skills identified in the previous
%   question, what are at least two approaches to acquiring the knowledge or
%   mastering the skill?  Of the identified approaches, which will each team
%   member pursue, and why did they make this choice?
% \end{enumerate}

\end{document}