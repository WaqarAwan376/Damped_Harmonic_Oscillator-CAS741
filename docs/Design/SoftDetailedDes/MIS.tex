\documentclass[12pt, titlepage]{article}

\usepackage{amsmath, mathtools}

\usepackage[round]{natbib}
\usepackage{amsfonts}
\usepackage{amssymb}
\usepackage{graphicx}
\usepackage{colortbl}
\usepackage{xr}
\usepackage{hyperref}
\usepackage{longtable}
\usepackage{xfrac}
\usepackage{tabularx}
\usepackage{float}
\usepackage{siunitx}
\usepackage{booktabs}
\usepackage{multirow}
\usepackage[section]{placeins}
\usepackage{caption}
\usepackage{fullpage}

\hypersetup{
bookmarks=true,     % show bookmarks bar?
colorlinks=true,       % false: boxed links; true: colored links
linkcolor=red,          % color of internal links (change box color with linkbordercolor)
citecolor=blue,      % color of links to bibliography
filecolor=magenta,  % color of file links
urlcolor=cyan          % color of external links
}

\usepackage{array}

\externaldocument{../../SRS/SRS}

%% Comments

\usepackage{color}

\newif\ifcomments\commentstrue %displays comments
%\newif\ifcomments\commentsfalse %so that comments do not display

\ifcomments
\newcommand{\authornote}[3]{\textcolor{#1}{[#3 ---#2]}}
\newcommand{\todo}[1]{\textcolor{red}{[TODO: #1]}}
\else
\newcommand{\authornote}[3]{}
\newcommand{\todo}[1]{}
\fi

\newcommand{\wss}[1]{\authornote{blue}{SS}{#1}} 
\newcommand{\plt}[1]{\authornote{magenta}{TPLT}{#1}} %For explanation of the template
\newcommand{\an}[1]{\authornote{cyan}{Author}{#1}}

%% Common Parts

\newcommand{\progname}{Damped Harmonic Ocsillator} % PUT YOUR PROGRAM NAME HERE
\newcommand{\authname}{
\\ Muhammad Waqar Ul Hassan Awan
} % AUTHOR NAMES                  

\usepackage{hyperref}
    \hypersetup{colorlinks=true, linkcolor=blue, citecolor=blue, filecolor=blue,
                urlcolor=blue, unicode=false}
    \urlstyle{same}
                                


\begin{document}

\title{Module Interface Specification for \progname{}}

\author{\authname}

\date{\today}

\maketitle

\pagenumbering{roman}

\section{Revision History}

\begin{tabularx}{\textwidth}{p{3cm}p{2cm}X}
\toprule {\bf Date} & {\bf Version} & {\bf Notes}\\
\midrule
18 Mar, 2024 & 1.0 & Initial MIS documentation\\
\bottomrule
\end{tabularx}

~\newpage

\section{Symbols, Abbreviations and Acronyms}

See SRS Documentation at \href{https://github.com/WaqarAwan376/Damped_Harmonic_Oscillator-CAS741/tree/main/docs/SRS}{SRS}

\newpage

\tableofcontents

\newpage

\pagenumbering{arabic}

\section{Introduction}

The following document details the Module Interface Specifications for a software project aimed at modeling and analyzing harmonic oscillators,
providing a tool for educational, research, and industrial applications to observe the effect
of damping on oscillating bodies.

Complementary documents include the System Requirement Specifications
and Module Guide.  The full documentation and implementation can be
found \href{https://github.com/WaqarAwan376/Damped_Harmonic_Oscillator-CAS741/tree/main}{here}.

\section{Notation}

The structure of the MIS for modules comes from Software Design, Automated Testing, and Maintenance: A Practical Approach,
with the addition that template modules have been adapted from
Fundamentals of Software Engineering.  The mathematical notation comes from Chapter 3 of 
Software Design, Automated Testing, and Maintenance: A Practical Approach.  For instance, the symbol := is used for a
multiple assignment statement and conditional rules follow the form $(c_1
\Rightarrow r_1 | c_2 \Rightarrow r_2 | ... | c_n \Rightarrow r_n )$.

The following table summarizes the primitive data types used by \progname. 

\begin{center}
\renewcommand{\arraystretch}{1.2}
\noindent 
\begin{tabular}{l l p{7.5cm}} 
\toprule 
\textbf{Data Type} & \textbf{Notation} & \textbf{Description}\\ 
\midrule
character & char & a single symbol or digit\\
real & $\mathbb{R}$ & any number in (-$\infty$, $\infty$)\\
\bottomrule
\end{tabular} 
\end{center}

\noindent
The specification of \progname \ uses some derived data types: sequences, strings, and
tuples. Sequences are lists filled with elements of the same data type. Strings
are sequences of characters. Tuples contain a list of values, potentially of
different types. In addition, \progname \ uses functions, which
are defined by the data types of their inputs and outputs. Local functions are
described by giving their type signature followed by their specification.

\section{Module Decomposition}

The following table is taken directly from the Module Guide document for this project.

\begin{table}[h!]
\centering
\begin{tabular}{p{0.3\textwidth} p{0.6\textwidth}}
\toprule
\textbf{Level 1} & \textbf{Level 2}\\
\midrule

{Hardware-Hiding} & Hardware-Hiding Module \\
\midrule

\multirow{5}{0.3\textwidth}{Behaviour-Hiding} & Data Input Module\\
& Parameter Input Module\\
& Calculation Engine Module\\
& Output Module\\
& User Interface Module\\ 
\midrule

\multirow{2}{0.3\textwidth}{Software Decision} & {Plotting and Visualization Module}\\
& ODE Solver Module\\
\bottomrule

\end{tabular}
\caption{Module Hierarchy}
\label{TblMH}
\end{table}

\newpage
~\newpage

\section{MIS of Data Input} \label{mDI}

Data input module Manages the input of raw data into the system, potentially including data validation and preprocessing.

\subsection{Module}

DataInput

\subsection{Uses}
N/A

\subsection{Syntax}

\subsubsection{Exported Constants}
None specified.

\subsubsection{Exported Access Programs}

\begin{center}
\begin{tabular}{p{4cm} p{4cm} p{4cm} p{4cm}}
\hline
\textbf{Name} & \textbf{In} & \textbf{Out} & \textbf{Exceptions} \\
\hline
receiveInputs & parameterValues: Map & Boolean & InvalidInputException \\
validateInputs & parameterValues: Map & Boolean & ValidationError \\
\hline
\end{tabular}
\end{center}

\subsection{Semantics}

\subsubsection{State Variables}
m: $\mathbb{R}$\\
k: $\mathbb{R}$\\
c: $\mathbb{R}$\\
$x_0$: $\mathbb{R}$\\
$v_0$: $\mathbb{R}$\\
g: $\mathbb{R}$\\
F(t): t: $\mathbb{R}$ → res: $\mathbb{R}$\\ % Function describing any time-dependent external force acting on the system
% State variables to track the oscillator's motion over time:
$x_i$: $\mathbb{R}$
$v_i$: $\mathbb{R}$
$a_i$: $\mathbb{R}$

\subsubsection{Environment Variables}

keyboard: Captures input from the system's keyboard device.

\subsubsection{Assumptions}

\begin{itemize}
  \item Inputs are provided in a structured format that the module can parse and validate.
  \item All necessary parameters (mass m, spring constant k, damping coefficient c, initial displacement x0, and initial velocity v0) are included in the input data.
\end{itemize}

\subsubsection{Access Routine Semantics}

\noindent receiveInputs(parameterValues):
\begin{itemize}
\item transition: Validates the provided parameterValues and, if valid, updates inputData.
\item output: Returns true if inputs are valid and successfully stored; otherwise, false.
\item exception: InvalidInputException: Triggered if parameterValues contains undefined or non-numeric values.
\end{itemize}

\noindent validateInputs(parameterValues):
\begin{itemize}
\item transition: N/A
\item output: Returns true if all values in parameterValues meet the physical and software constraints; otherwise, false.
\item exception: ValidationError: Triggered if any value in parameterValues does not adhere to the constraints defined in Section 4.2.6 of the SRS document.
\end{itemize}

\subsubsection{Local Functions}

\begin{itemize}
  \item isNumeric: Checks if a given input is a numeric value.
  \item meetsConstraints: Verifies if a given value adheres to specified constraints for each parameter (as defined in SRS Section 4.2.6).
\end{itemize}


% ##########

\section{MIS of Parameter Input Module} \label{mPI}

Parameter input module Processes and validates the parameters specific to the damped harmonic oscillator
simulation, such as mass, spring constant, damping coefficients, etc.

\subsection{Module}
ParameterInput

\subsection{Uses}
Data Input Module

\subsection{Syntax}

\subsubsection{Exported Constants}
None specified.

\subsubsection{Exported Access Programs}

\begin{center}
\begin{tabular}{p{3cm} p{5cm} p{2cm} p{5cm}}
\hline
\textbf{Name} & \textbf{In} & \textbf{Out} & \textbf{Exceptions} \\
\hline
setParameter & paramName:String,value: Real & Boolean & InvalidParameterException \\
getParameter & paramName:String & Real & ParameterNotFoundException \\
\hline
\end{tabular}
\end{center}

\subsection{Semantics}

\subsubsection{State Variables}
m: $\mathbb{R}$\\
k: $\mathbb{R}$\\
c: $\mathbb{R}$\\
$x_0$: $\mathbb{R}$\\
$v_0$: $\mathbb{R}$\\
F(t): t: $\mathbb{R}$ → res: $\mathbb{R}$ % Function describing any time-dependent external force acting on the system

\subsubsection{Environment Variables}

N/A

\subsubsection{Assumptions}

\begin{itemize}
  \item The module is called after all necessary parameters are initialized to default values.
  \item Parameters passed to setParameter are within the valid range as defined in the SRS and are the correct type (Real numbers for most parameters).
\end{itemize}

\subsubsection{Access Routine Semantics}

\noindent setParameter(paramName, value):
\begin{itemize}
\item transition: Updates the state variable corresponding to paramName with value.
\item output: Returns true if the parameter is successfully updated; otherwise, false.
\item exception: InvalidParameterException: Triggered if value is out of the acceptable range or paramName does not correspond to a valid parameter name.
\end{itemize}

\noindent getParameter(paramName):
\begin{itemize}
\item transition: N/A
\item output: Returns the value of the state variable corresponding to paramName.
\item exception: ParameterNotFoundException: Triggered if paramName does not correspond to a valid parameter name.
\end{itemize}

\subsubsection{Local Functions}

\begin{itemize}
  \item validateParameter: Ensures that a given parameter value is within the acceptable range and type for the parameter.
\end{itemize}


% ##########

\section{MIS of Calculation Engine Module} \label{mCE}

Calculation engine module Performs the core calculations based on input parameters and models, generating
the dynamics of the oscillator over time.

\subsection{Module}
CalcEngine

\subsection{Uses}
Data Input, Parameter Input modules

\subsection{Syntax}

\subsubsection{Exported Constants}
None specified.

\subsubsection{Exported Access Programs}

\begin{center}
\begin{tabular}{p{4cm} p{4cm} p{3cm} p{5cm}}
\hline
\textbf{Name} & \textbf{In} & \textbf{Out} & \textbf{Exceptions} \\
\hline
calculateMotion & time: Real & MotionResults & CalculationException \\
getResult & paramName: String & Real & ResultNotFoundException \\
\hline
\end{tabular}
\end{center}

\subsection{Semantics}

\subsubsection{State Variables}
timeSeries: Array$<\mathbb{R}>$\\ % Time points at which the motion is calculated.
displacementSeries: Array$<\mathbb{R}>$\\ % Displacement values corresponding to each time point.
velocitySeries: Array$<\mathbb{R}>$\\ % Velocity values corresponding to each time point.
accelerationSeries: Array$<\mathbb{R}>$ % Acceleration values calculated for each time point.

\subsubsection{Environment Variables}

N/A

\subsubsection{Assumptions}

\begin{itemize}
  \item Parameters required for the calculations (m, k, c, x0, v0) have been correctly set and validated before invoking calculateMotion.
  \item The time parameter for the calculateMotion method is within the simulation time range defined by the user.
\end{itemize}

\subsubsection{Access Routine Semantics}

\noindent calculateMotion(time):
\begin{itemize}
\item transition: Uses the current set parameters (m, k, c, x0, v0) and external forces, if any, to calculate the motion of the damped harmonic oscillator at the specified time point. Updates timeSeries, displacementSeries, velocitySeries, and accelerationSeries with the calculated values.
\item output: MotionResults, a structure containing time, displacement, velocity, and acceleration at the specified time.
\item exception: CalculationException: Triggered if the calculation fails due to invalid parameters or numerical errors.
\end{itemize}

\noindent getResult(paramName):
\begin{itemize}
\item transition: N/A
\item output: Returns the series (Array$<\mathbb{R}>$) corresponding to the requested parameter name ("displacement", "velocity", or "acceleration").
\item exception: ResultNotFoundException: Triggered if paramName does not correspond to any calculated result series.
\end{itemize}

\subsubsection{Local Functions}

\begin{itemize}
  \item solveDifferentialEquation: Applies the appropriate numerical method to solve the differential equation governing the damped harmonic oscillator's motion based on the parameters and external forces.
\end{itemize}


% ##########

\section{MIS of Output Module} \label{mOM}

Output module transforms calculation results into various output formats (e.g., graphical, tex-
tual) for user interpretation.

\subsection{Module}
Output

\subsection{Uses}
Calculation Engine, User Interface modules

\subsection{Syntax}

\subsubsection{Exported Constants}
None specified.

\subsubsection{Exported Access Programs}

\begin{center}
\begin{tabular}{p{4cm} p{4cm} p{3cm} p{5cm}}
\hline
\textbf{Name} & \textbf{In} & \textbf{Out} & \textbf{Exceptions} \\
\hline
displayResults & results:MotionResults,outputType: String & Boolean & DisplayException \\
\hline
\end{tabular}
\end{center}

\subsection{Semantics}

\subsubsection{State Variables}
N/A

\subsubsection{Environment Variables}
\begin{itemize}
  \item \textbf{screen:} A device or interface for displaying results to the user.
\end{itemize}

\subsubsection{Assumptions}

\begin{itemize}
  \item The results object contains valid motion data (displacement, velocity, acceleration) for the damped harmonic oscillator.
  \item The outputType specifies the format in which the user prefers to view the results (e.g., "graph", "table").
\end{itemize}

\subsubsection{Access Routine Semantics}

\noindent displayResults(results, outputType):
\begin{itemize}
\item transition: N/A
\item output: Displays the motion results in the specified format (outputType) on the screen.
\item exception: DisplayException: Triggered if the specified outputType is not supported or if there is an error in rendering the results.
\end{itemize}

\subsubsection{Local Functions}

\begin{itemize}
  \item \textbf{formatResults:} Formats the motion results into the specified output type (e.g., converting data into a graphical representation or a table format for display purposes).
\end{itemize}


% ##########

\section{MIS of User Interface Module} \label{mUIM}

User Interface module provides the graphical or command-line interface through which users interact
with the software, including inputting parameters and viewing results.

\subsection{Module}
UserInterface

\subsection{Uses}
N/A

\subsection{Syntax}

\subsubsection{Exported Constants}
None specified.

\subsubsection{Exported Access Programs}

\begin{center}
\begin{tabular}{p{4cm} p{4cm} p{3cm} p{5cm}}
\hline
\textbf{Name} & \textbf{In} & \textbf{Out} & \textbf{Exceptions} \\
\hline
displayMainMenu & None & None & UIException \\
getInput & prompt: String & String & InputFormatException \\
displayResults & results: MotionResults & None & DisplayException \\
displayError & message: String & None & None \\
confirmAction & confirmationMessage: String & Boolean & None \\
\hline
\end{tabular}
\end{center}

\subsection{Semantics}

\subsubsection{State Variables}
N/A

\subsubsection{Environment Variables}
\begin{itemize}
  \item \textbf{keyboard:} Interface for capturing user input.
  \item \textbf{screen:} Interface for displaying information to the user.
\end{itemize}

\subsubsection{Assumptions}

\begin{itemize}
  \item Users interact with the software through a graphical or command-line interface that provides clear prompts and displays.
  \item Input errors and exceptions are handled gracefully, informing the user of the issue without causing crashes or unhandled errors.
\end{itemize}

\subsubsection{Access Routine Semantics}

\noindent displayMainMenu():
\begin{itemize}
\item transition: N/A
\item output: Displays the main menu options to the screen, allowing users to choose actions like parameter entry, simulation start, or results viewing.
\item exception: UIException: Triggered if there is an issue displaying the main menu.
\end{itemize}

\noindent getInput(prompt):
\begin{itemize}
\item transition: N/A
\item output: Returns the user input collected in response to the displayed prompt.
\item exception: InputFormatException: Triggered if the input provided by the user does not match the expected format for the prompt.
\end{itemize}

\noindent displayResults(results):
\begin{itemize}
\item transition: N/A
\item output: Formats and displays the simulation results contained in results to the screen.
\item exception: DisplayException: Triggered if there is an issue displaying the results.
\end{itemize}

\noindent displayError(message):
\begin{itemize}
\item transition: N/A
\item output: Displays an error message to the screen, informing the user of any issues encountered during operation.
\item exception: N/A
\end{itemize}

\noindent confirmAction(confirmationMessage):
\begin{itemize}
\item transition: N/A
\item output: Displays a confirmationMessage to the user and returns true if the user confirms the action; otherwise, returns false.
\item exception: N/A
\end{itemize}

\subsubsection{Local Functions}
N/A


% ##########

\section{MIS of Plotting and Visualization Module} \label{mPVM}

Plotting and Visualization module creates charts, graphs, and animations to visually depict the behavior of the
damped harmonic oscillator based on the simulation outcomes.

\subsection{Module}
PlotVis

\subsection{Uses}
Output Module

\subsection{Syntax}

\subsubsection{Exported Constants}
None specified.

\subsubsection{Exported Access Programs}

\begin{center}
\begin{tabular}{p{4cm} p{4cm} p{3cm} p{5cm}}
\hline
\textbf{Name} & \textbf{In} & \textbf{Out} & \textbf{Exceptions} \\
\hline
plotResults & results:MotionResults, plotType: String & Boolean & PlottingException \\
\hline
\end{tabular}
\end{center}

\subsection{Semantics}

\subsubsection{State Variables}
N/A

\subsubsection{Environment Variables}
\begin{itemize}
  \item \textbf{displayDevice:} A device or interface for visual output, such as a monitor for displaying plots.
\end{itemize}

\subsubsection{Assumptions}

\begin{itemize}
  \item The results provided contain valid and complete data necessary for plotting and report generation.
  \item The plotType specifies the desired type of plot (e.g., "displacement vs. time", "velocity vs. time").
\end{itemize}

\subsubsection{Access Routine Semantics}

\noindent plotResults(results, plotType):
\begin{itemize}
\item transition: Generates and displays a plot based on the results and the specified plotType.
\item output: Returns true if the plot is successfully generated and displayed; otherwise, false.
\item exception: PlottingException: Triggered if there is an error in generating or displaying the plot, such as an unsupported plotType or a plotting library error.
\end{itemize}

\subsubsection{Local Functions}

\begin{itemize}
  \item \textbf{formatResultsForPlotting:} Formats the motion results into a structure suitable for plotting based on the selected plotType.
\end{itemize}


% ##########

\section{MIS of ODE Solver Module} \label{mPVM}

ODE solver module provides the computational backbone for accurately solving the equations of mo-
tion under various conditions and damping models.

\subsection{Module}
ODESolver

\subsection{Uses}
Calculation Engine module

\subsection{Syntax}

\subsubsection{Exported Constants}
None specified.

\subsubsection{Exported Access Programs}

\begin{center}
\begin{tabular}{p{2cm} p{5cm} p{2cm} p{5cm}}
\hline
\textbf{Name} & \textbf{In} & \textbf{Out} & \textbf{Exceptions} \\
\hline
solveODE & equation: ODE, initialConditions: Conditions, timeSpan: TimeSpan & Solution & ODESolverException \\
\hline
\end{tabular}
\end{center}

\subsection{Semantics}

\subsubsection{State Variables}
\begin{itemize}
  \item \textbf{equation:} ODE - The ordinary differential equation to be solved, representing the motion of the damped harmonic oscillator.
  \item \textbf{initialConditions:} Conditions - Initial state of the system, including initial displacement and velocity.
  \item \textbf{timeSpan:} TimeSpan - The range of time over which to solve the ODE.
\end{itemize}

\subsubsection{Environment Variables}
N/A

\subsubsection{Assumptions}

\begin{itemize}
  \item The ODE provided correctly models the damped harmonic oscillator based on parameters set in the ParameterInput module.
  \item Initial conditions and time span are appropriate for the problem being solved and have been validated before calling solveODE.
\end{itemize}

\subsubsection{Access Routine Semantics}

\noindent solveODE(equation, initialConditions, timeSpan):
\begin{itemize}
\item transition: Uses numerical methods to solve the provided equation over the specified timeSpan starting from initialConditions.
\item output: Solution, an object containing the time series and corresponding solution values (displacement and velocity) for the ODE.
\item exception: ODESolverException: Triggered if the solver fails to converge to a solution or encounters numerical instabilities.
\end{itemize}

\subsubsection{Local Functions}

\begin{itemize}
  \item \textbf{numericalIntegrationMethod:} A numerical solver that approximates the solution of the ODE based on the input equation, initial conditions, and time span.
\end{itemize}


\newpage

\bibliographystyle {plainnat}
\bibliography {../../../refs/References}

\newpage

\section{Appendix} \label{Appendix}

\begin{longtable}{l p{12cm}}
  \toprule
  \textbf{Message ID} & \textbf{Error Message} \\
  \midrule
  FileNotFound & Error: The specified file could not be found. \\
  InvalidParameter & Error: One or more parameters provided are invalid or out of range. \\
  ParameterNotFound& Error: The requested parameter does not exist.\\
  CalculationError& Error: An error occurred during the calculation process.\\
  DisplayError& Error: Failed to display the requested information.\\
  SaveError& Error: The results could not be saved to the specified location.\\
  InputFormatException& Error: The provided input does not match the expected format.\\
  PlottingError& Error: An error occurred while generating the plot.\\
  ReportGenerationError& Error: Failed to generate the report in the specified format.\\
  ODESolverError& Error: Failed to solve the ODE due to numerical instabilities or convergence issues.\\
  UIError& Error: An unexpected error occurred in the User Interface.\\
  \bottomrule
  \caption{Possible Exceptions} \\
\end{longtable}


\end{document}